\section{Error on prediction}
Running ahead on the actual measurements, let's estimate how accurate our
predictions will be. Let's label the vertical spatial accuracy,
i.e how accurately we know the height of these channels, $\delta z$\footnote{meaning if we have a measurement $z_i$ the true value is within $z_i
\pm \delta z$ with 95\% certainty}. And the timely accuracy, or how accurate the
vpol timing characteristics are, $\delta t$. If we want to fit n, we first
minimize the summed correlation function to construct the plane wave. One
correlation function is given by 
\begin{equation}
	Correlation(\theta) := \Delta t - \Delta t' = \Delta t
	- \frac{\cos\theta \Delta z}{v}
\end{equation}
Say the minimum occurs at $Correlation(\theta) = Correlation(\theta_{min}) := \mathcal{C}$, we can then re-write 
this equation for n:
\begin{equation}
  n = \frac{c(\Delta t - \mathcal{C})}{\cos \theta_{min} \Delta z} = \frac{c}{\cos{\theta_{min}}}\left[\frac{\Delta t}{\Delta z} - \frac{\mathcal{C}}{\Delta z}\right]
\end{equation}
The $\Delta z$ is the vertical distance between two channels and thus has an error of $2\delta z$ and $\Delta t$ is the time
difference between two channels, implying an error $2\delta t$. 
Let's look at the two terms seperately (assuming the error on $\theta_{min}$ to be negligable),
the first quotient $\phi_1 := \Delta t/\Delta z$ has a variance \cite{grabe2005measurement} of
\begin{equation}
	s_{\phi_1}^2 = \frac{1}{{\Delta z}^2}s_{\Delta t}^2 - 2 \frac{{\Delta t}}{{\Delta z}^3}s_{\Delta t \Delta z} + \frac{{\Delta t}^2}{{\Delta z}^4}s_{\Delta z}^2
\end{equation}
Assuming $\Delta t$ and $\Delta z$ to be independent:
\begin{align}
	s_{\phi_1}^2 &= \frac{1}{{\Delta z}^2}s_{\Delta t}^2 + \frac{{\Delta t}^2}{{\Delta z}^4}s_{\Delta z}^2\\
	&= 4\left(\frac{1}{{\Delta z}^2}{\delta t}^2 + \frac{{\Delta t}^2}{{\Delta z}^4}{\delta z}^2\right)
\end{align}
And the second term ($\phi_2 := -\mathcal{C}/\Delta z$):
\begin{align}
	s_{\phi_2}^2 &= \frac{\mathcal{C}^2}{{\Delta z}^2}s_{\Delta t}^2\\
		     &= 4\frac{\mathcal{C}^2}{{\Delta z}^2}{\delta t}^2\\
\end{align}
Our squared variance on the index of refraction is thus (neglecting unknown systematic errors):
\begin{equation}
	\delta n^2 =: s_n^2 = 4\left(\frac{c}{\cos{\theta_{min}}}\right)^2 \left( \frac{1}{{\Delta z}^2}{\delta t}^2 + \frac{{\Delta t}^2}{{\Delta z}^4}{\delta z}^2 + 
\frac{\mathcal{C}^2}{{\Delta z}^2}{\delta t}^2\right)
\end{equation}
If we have more than 2 detectors, say N then the uncertainty on the fit can be assumed to be the
RMS of the individual uncertainties:
\begin{equation}
  \delta n =\sqrt{\sum_{i=0}^N \delta n _i^2}
\end{equation}
Let's assume $\epsilon$ to be an absolute error.  Now due to this inherent
inaccuracy, the "global" uncertainty on n also has an additional error of $\pm \epsilon(\vec{r})n$ with
$\vec{r}$ the position of the balloon.  Our final error on n is thus:
\begin{equation}
  \delta n(\vec{r})=  \epsilon(\vec{r})n + \sqrt{\sum_{i=0}^N 4\left(\frac{c}{\cos{\theta_{min}}}\right)^2 \left( \frac{1}{{\Delta z_i}^2}{\delta t}^2 + \frac{{\Delta t_i}^2}{{\Delta z_i}^4}{\delta z}^2 + 
  \frac{\mathcal{C}_i^2}{{\Delta z_i}^2}{\delta t}^2\right)}
  \label{eqn:total error}
\end{equation}
If we \textbf{assume the $\epsilon$ to overestimate the index of refraction the same way in real
life as in the simulation} however, our estimated $n$ can be corrected as
\begin{equation}
  n_{\text{corrected}} (\vec{r})= \frac{n(\vec{r})}{\epsilon(\vec{r}) + 1}
\end{equation}
and our error becomes only the second part of equation \ref{eqn:total error}.
As the error on the position of the channels is not yet fully known most of
this section mainly serves as a future reference when people continue this
work, however it can be safely assumed that the timely error will be of much
higher importance than the positional error, as we'll get to in section (???).  

The error on timing can be estimated, contrary to the positional error, from the
sampling rate, as if we have a sampling rate of e.g 3.2GHz then the antenna
will take a measurement every 
\begin{equation}
	\delta t = \frac{1}{3.2\text{GHz}} = 0.3125\text{ns}
\end{equation}
So we can predict a measurement of the index of refraction, assuming $\epsilon$
to be a correction and $\delta z \ll$, to have an error of
\begin{align}
	\delta n(\vec{r})_{\text{corrected}}&=  (1+\epsilon)\times\sqrt{\sum_{i=0}^N 4\left(\frac{c}{\cos{\theta_{min}}}\right)^2 \left( \frac{1}{{\Delta z_i}^2}{\delta t}^2 +
  \frac{\mathcal{C}_i^2}{{\Delta z_i}^2}{\delta t}^2\right) }\\
					    &=  2(1+\epsilon){\delta t} \times\sqrt{\sum_{i=0}^N \left(\frac{c}{\cos{\theta_{min}}}\right)^2 \left( \frac{1}{{\Delta z_i}^2}+
  \frac{\mathcal{C}_i^2}{{\Delta z_i}^2}\right) }\\
					    &=  2(1+\epsilon){\delta t} \times\sqrt{\sum_{i=0}^N \left(\frac{c}{\Delta z_i\cos{\theta_{min}}}\right)^2 \left(1 +
  \mathcal{C}_i^2\right)}
  \label{eqn:errorcorr}
\end{align}
And if we only consider 2 detectors (which we will in most cases) this reduces
to
\begin{equation}
	\delta n(\vec{r})_{\text{corrected}} = 2(1+\epsilon)\times c\left(\frac{\delta t}{\Delta z}\right) \times\left[\frac{\sqrt{1 +
	\mathcal{C}^2}}{\cos{\theta_{min}}}\right]
\end{equation}
Note again that $\Delta z$ is the \textbf{difference} in height between the two
detectors and $\delta t$ is the \textbf{accuracy} in timing (which gets
determined by the sample rate of the antenna).

\newpage

