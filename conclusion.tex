\chapter*{Conclusion}
\addcontentsline{toc}{chapter}{Conclusion}
The hybrid ray tracer developed over the course of this thesis and reported in
chapter \ref{chapter:hybrid} was a succes, not only is it more accurate than
it's predecessors but it's also faster.  The accuracy of the ray tracer also
makes it perfectly suitable for very fine precision simulations as used within
the Balloon radio wave measurements.

The use of weather balloons to estimate the index of refraction as
was reported in \ref{chap:WB} was also done succesfully even though
it yielded quite unexpected results, not only were the index of
refraction measured via this method for channels 5-7 deviating from the model but
also deviating from the theoretically expected indices using the
measured density in combination with Schytts equation
\begin{equation}
	n(z) = 1+ 0.78\rho/\rho_0
\end{equation}
moreso lying somewhere inbetween. A conversion using the inverse of
Schytts equation to arrive at the density makes this especially
clear as shown on figure \ref{fig:BalloonDensity6And7}.
These results are thus probably incorrect as the plane wave
reconstruction cannot be used for channels lying that far 
away from eachother. It can be used however to estimate
the index of refraction at the phased array.

As there are a multitude of events (as can be found in appendices
\ref{app:5Deg} and \ref{app:10Deg} but only finite time to analyse
each and every one, this code is made public
\href{https://github.com/arthuradriaens-code/projects-mt.git}{here}
as to make it easy to improve on this work.  Especially interesting
might be events where $\epsilon$ is negligable (the balloon is
really close) at the phased array and as such is the plane wave
reconstruction a near perfect representation of the actual wave.

