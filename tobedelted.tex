\chapter{To be deleted}
Now going through the full calculation (i.e minimizing the correlation function
for different n given the time difference and finding the minimal distance
between reconstruction and balloon position)\footnote{Note that we used Snell's
law} we get that the index for refraction at a depth of 47.714m is 1.6061,
which is a fairly good estimate as, looking at the measurements depicted in
figure \ref{fig:DensityMeasurements} a depth of 47.7m would correspond to a
density of about $710 kg/m^3$, using Schytts equation this would thus give an
index of refraction of 1.603.  Now using the positional data we can calculate
the relative accuracy $\epsilon$ which comes to be about $0.39\%$. Our final
answer is thus:
\begin{table}[h]
    \centering
    \begin{tabular}{c|c|c|c|c|c}
      depth (m)& station number & run used & event\_id used & channels used & n\\
      \hline
      -47.714 & 23 & 691 & 489 & 6,7 & $1.606 \pm 0.006$
    \end{tabular}
\end{table}\\
Which has the expected answer within the margin of error. We are aware that
this event also shows a peak in the frequency spectrum for channel 5 but the
data deviates slightly from an AM signal and thus didn't seem to be usable due
to the difficulty of fitting it. 
\newpage
\section{Fitting the index: deep components}
%\subsection{Station 24 run 646 event 373 Channels 0,1,2,3}
%Going through the full calculation again, i.e getting the relative balloon
%position and detector positions, fitting the AM function and thus finding the
%time offsets, minimizing the correlation function for every index of refraction
%n whilst recording the difference between the path and the weather balloon
%location and finally returning the index of refraction for which this
%difference is the smallest; we get as an index of refraction from fit at a
%depth of -96.1265m : 1.7648612972259445, if we now calculate the theoretical
%value (for a greenland simple ice model) of the \textit{relative accuracy}
%$\epsilon$ we get $3.369\%$. This thus implies a value of
%\begin{equation}
%  n = 1.765 \pm 0.059
%\end{equation}
%It is quite unfortunate that the uncertainty is that large.
%Now how does this compare to the expected value from density measurements?
%Looking at the density measurements shown in figure \ref{fig:DensityMeasurements} 
%we expect a density of about 860 $kg/m^3$ at a depth of 96.13m, using
%Schytt's equation with $\rho_0 = 917$ this implies an index of refraction of about 1.73
%which is within our confidence interval.

\subsection{Station 23 run 800 event 1867 Channels 0 and 3}
Using guesses $\phi_2^0 = -786.91$ \& $\phi_2^3 = -782$, traveltimes
[4822.41659697 4806.73065199] are found which have nearly the same difference
and size as the simulated [4822.16,4805.80]. The minimal correlation was found to be
$0.334$ns and yielded 1.7475 as index of refraction at a depth of -93.231m.
Epsilon is calculated to be 4.865\% for this Balloon position, our final result is 
thus:
\begin{table}[h]
    \centering
    \begin{tabular}{c|c|c|c|c|c}
      depth (m)& station number & run used & event\_id used & channels used & n\\
      \hline
      -93.231& 23 & 800 & 1867 & 0,3 & $1.748 \pm 0.085$
    \end{tabular}
\end{table}\\
From analysis of the density we expect an index of refraction of 1.73 which
lies inside our confidence interval.

\section{Special test: Channels 7 and 13}
\begin{table}[h]
    \centering
    \begin{tabular}{c|c|c|c|c|c}
      depth (m)& station number & run used & event\_id used & channels used & n\\
      \hline
      -19.11& 23 & 691 & 489 & 7,13 & $1.3777 \pm 0.117$
    \end{tabular}
\end{table}
This test isn't to be used as the error is too large but it might be useful
to know that a plane wave reconstruction with one of the nearest deep components and
a surface component is possible.

