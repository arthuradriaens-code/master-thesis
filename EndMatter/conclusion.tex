\chapter*{Conclusion}
\addcontentsline{toc}{chapter}{Conclusion}
The hybrid ray tracer developed over the course of this thesis and reported in
chapter \ref{chapter:hybrid} was a success, not only is it more accurate than
it's predecessors, as shown in figures \ref{fig:acchyb} and \ref{fig:accit},
 but it's also faster by about 34\%.  The accuracy of the ray tracer also
makes it perfectly suitable for very fine precision simulations as used within
the Balloon radio wave measurements.

The use of weather balloons to estimate the index of refraction as
was reported in \ref{chap:WB} was also done successfully 
yielding the following data:
\begin{center}
\begin{tabular}{||c c c c c c||}
 \hline
 Depth (m) & Station id & channels & Run:Event & n$_\text{exponential}$ & n$_\text{fit}$\\ [0.5ex]
 \hline\hline
 -94.518 & 11 & 0\&2\&3 & 1034:12397 & 1.7397 & 1.7045 $\pm$ 0.006 \\
 \hline
\end{tabular}
\end{center}
And visually shown as the data in blue on figure \ref{fig:IndexVSDepth.pdf},
it clearly shows a big discrepancy with the exponential model.
We reason that this clearly shows that there's a need to investigate
the index-depth relation further as the wave propagation, and thus
neutrino reconstruction, are heavily dependent on the ice model.

Note that there was only one event in the phased array that was analysed, this 
is because the occasions of when a balloon passes close enough to a detector
and when the signal is measurable in the deep array are very few.

As there are a multitude of events (as can be found in appendices
\ref{app:5Deg} and \ref{app:10Deg} but only finite time to analyze each and
every one, this code is also made public
\href{https://github.com/arthuradriaens-code/projects-mt.git}{here} as to make
it easy to improve on this work.  Especially interesting might be events where
$\epsilon$ is negligible (the balloon is really close) at the phased array and
as such is the plane wave reconstruction a near perfect representation of the
actual wave.
\newpage
As this method seems to be a viable way of measuring the index of refraction,
it might be a good idea to have a more controllable radio wave source fly
closer to the detectors to make more accurate measurements, e.g an
autonomous\footnote{as to not have it need a radio controller, causing RF
interference on top of the one coming from the engine and also as autonomous
GNSS positioning is always more accurate than human steering} drone with an
antenna strapped to it or just some form of stationed antenna.  

It might also be a good idea to make the signal easier to fit, as the currently
used signal has quite a short wavelength (0.403GHz) we have to guess when the 
signal roughly arrives. If we have, say an AM signal, we wouldn't have to assume
the rough arrival time of the signal, thus reducing the possible error on timing.

In the ideal case, as has been done in the South Polar firn\cite{kravchenko_besson_meyers_2004}, we'd want some immobile close source and a phased array which we'll
move down every couple of measurements, thus obtaining an index-depth profile
using direct measurements. This might be carried out e.g during deployment
of a new station's power string.
