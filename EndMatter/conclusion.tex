\chapter*{Conclusion}
\addcontentsline{toc}{chapter}{Conclusion}
The hybrid ray tracer developed over the course of this thesis and reported in
chapter \ref{chapter:hybrid} was a success, not only is it more accurate than
it's predecessors but it's also faster.  The accuracy of the ray tracer also
makes it perfectly suitable for very fine precision simulations as used within
the Balloon radio wave measurements.

The use of weather balloons to estimate the index of refraction as
was reported in \ref{chap:WB} was also done successfully even though
it yielded quite unexpected results as shown below:
\begin{center}
\begin{tabular}{||c c c c c c c||}
 \hline
 Depth (m) & Station id & channels & Run:Event & n$_\text{exponential}$ & $\epsilon$ & n$_\text{fitted}$\\ [0.5ex]
 \hline\hline
-48.155 & 21 & 6\&7 & 1441:117 & 1.6400 & -0.02\% & 1.6310 $\pm$ 0.001 \\
 -58.38 & 21 & 5\&7 & 1441:117 & 1.6736 & -0.23\% & 1.6688 $\pm$ 0.0006 \\
 -68.2 & 21 & 5\&6 & 1441:117 & 1.6983 & 0.02\% & 1.6916 $\pm$ 0.0003 \\
 -93.237 & 23 & phased array & 800:1867 & 1.7383 & 4.6166\% & 1.7393 $\pm$ 0.0013 \\
 -94.518 & 11 & 0\&2\&3 & 1034:12397 & 1.7397 & 0.3642\% & 1.6983 $\pm$ 0.0003 \\
 \hline
\end{tabular}
\end{center}
Herein the most accurate and trustworthy measurements is the last one which
also shows the biggest discrepancy with the exponential model. Further
analysis is thus needed.

As there are a multitude of events (as can be found in appendices
\ref{app:5Deg} and \ref{app:10Deg} but only finite time to analyze each and
every one, this code is made public
\href{https://github.com/arthuradriaens-code/projects-mt.git}{here} as to make
it easy to improve on this work.  Especially interesting might be events where
$\epsilon$ is negligible (the balloon is really close) at the phased array and
as such is the plane wave reconstruction a near perfect representation of the
actual wave.

