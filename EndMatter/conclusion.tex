\chapter*{Conclusion}
\addcontentsline{toc}{chapter}{Conclusion}
The hybrid ray tracer developed over the course of this thesis and reported in
chapter \ref{chapter:hybrid} was a success, not only is it more accurate than
its predecessors, as shown in figures \ref{fig:acchyb} and \ref{fig:accit},
 but it's also faster by about 34\%.  The accuracy of the ray tracer also
makes it perfectly suitable for very fine precision simulations as used within
the Balloon radio wave systematic error prediction (section \ref{sec:feasible}).

The use of weather balloons to estimate the index of refraction as
was reported in \ref{chap:WB} was also done successfully 
yielding the following data:
\begin{center}
\begin{tabular}{||c c c c c c||}
 \hline
 Depth (m) & Station id & channels & Run:Event & n$_\text{exponential}$ & n$_\text{fit}$\\ [0.5ex]
 \hline\hline
 -48.155 & 21 & 6\&7 & 1441:117 & 1.6400 & 1.632 $\pm$ 0.00128 $\pm$ 0.00472\\
 -58.38 & 21 & 5\&7 & 1441:117 & 1.6736 & 1.66632 $\pm$ 0.00226 $\pm$ 0.00233 \\
 -58.24 & 21 & 5\&6\&7 & 1441:117 & 1.67321 & 1.66553 $\pm$ 0.00167 $\pm$ 0.00350 \\
 -68.2 & 21 & 5\&6 & 1441:117 & 1.6983 & 1.69329 $\pm$0.00199$\pm$0.00460 \\
 -93.865 & 21 & 0\&1\&2\&3 & 1441:117 & 1.73896 & 1.71366$\pm$0.00287$\pm$0.05641\\
 -94.518 & 11 & 0\&2\&3 & 1034:12397 & 1.7397 & 1.724 $\pm$ 0.014 $\pm$ 0.047 \\
 \hline
\end{tabular}
\end{center}
It can be argued, as the fits are consistently below the exponential model,
that this data shows a shortcoming of the model.  Because of this we reason
that there's a need to investigate the index-depth relation further through 
experiments.

As there are a multitude of events (as can be found in appendices
\ref{app:5Deg} and \ref{app:10Deg}) but only finite time to analyze each and
every one, this code is also made public \cite{projects-mt} as to make
it accesible to improve on this work.  

As this method seems to be a viable way of measuring the index of refraction,
we'd want to use this method to determine a full index-depth profile. One major
drawback of using balloons and the detector however is that we can only measure
the index inbetween close lying channels.  To actually measure the full
index-depth profile then we'd ideally want some immobile close source and a
phased array which we'll move down every couple of measurements, as has been
done in the South Polar firn\cite{kravchenko_besson_meyers_2004}.  This might
be carried out e.g during deployment of a new station's power string.
