\chapter*{Introduction}
\addcontentsline{toc}{chapter}{Introduction}
\pagenumbering{arabic}
Outside our earth various kinds of events take place which we wish to 
observe: Black hole outbursts, supernovae, cosmic jets, ...
These events produce various kinds of messengers which are useful in
detecting them: Gravitational waves, gamma rays, protons,...
But one particle within this set of particles is quite unique and
the subject of our study: The neutrino. 

The neutrino is unique in that it points back to the event itself.
Due to the neutrino not having any charge, nearly no mass and 
interacting weakly it doesn't get bend or absorbed and re-emitted 
on it's way to us unlike say the proton. This means that if we observe
a neutrino back here on earth it's very likely that the direction we observe
it in, is the direction it came from.

We wish to detect a particular type of neutrinos: The ultra high energy (UHE)
neutrino.  There have been lots of neutrino detectors around but none of them
have been able to observe \textit{cosmogenic neutrinos} which would live past
the 3 PeV energy range, this is probably caused by the exponentially falling
flux with increasing energy, implying that a really big detector volume is
required to detect neutrinos with such high energies. 

Due to cost requirements, it was necessary to work in the radio regime.  As
much as we'd like a really big detector working in the visible spectrum, it
would cost way to much. As neutrinos can produce radio waves upon interacting
in ice through the Askaryan effect and as radio waves can travel for way longer
distances in there before interacting than visible light, it was previously
descided in experiments like ARA and ARIANNA to detect on the principle of
radio waves. The Radio Neutrino Observatory in Greenland or RNO-G wich is
currently under construction and the subject of this thesis builds on the
knowledge of these two experiments to make a quite complex detector which
should be capable of detecting UHE neutrinos.

The ice properties have an impact on how radiowaves propagate. As mentioned before, RNO-G is a detector built
in the Greenland icecap. As the radiowaves get produced in the ice we need to figure out how
they propagate towards our detector. An important part in figuring out how they propagate is 
the local index of refraction which seems to be linearly related to the density of the ice. 
The density of the ice seems to vary continuously with depth, a function describing
this overall relation of the index of refraction with the depth is called an \textit{ice model}
and it is crucial for future studies to understand this ice model.

It has become apparent that the ice model seems to deviate from the
theoretically expected single exponential density expectation.  In this thesis
a new algorithm will be developed which makes it possible to work with more
complex ice models and the verification of the shortcomings of the exponential
model, which is now the only model that gets used, will be layed out.

