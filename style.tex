% Optional: margins and spacing
%-------------------------------
% Uncomment and adjust to change the default values set by the template
% Note: the defaults are suggested values by Ghent University
%\geometry{bottom=2.5cm,top=2.5cm,left=3cm,right=2cm} 
%\renewcommand{\baselinestretch}{1.15} % line spacing
\setlength{\headheight}{13.59999pt}

% Python code
%---------------------------------------------
\usepackage{tcolorbox}
\tcbuselibrary{minted,breakable,xparse,skins}

\definecolor{bg}{gray}{0.95}
\DeclareTCBListing{mintedbox}{O{}m!O{}}{%
  breakable=true,
  listing engine=minted,
  listing only,
  minted language=#2,
  minted style=default,
  minted options={%
    linenos,
    gobble=0,
    breaklines=true,
    breakafter=,,
    fontsize=\small,
    numbersep=8pt,
    #1},
  boxsep=0pt,
  left skip=0pt,
  right skip=0pt,
  left=25pt,
  right=0pt,
  top=3pt, bottom=3pt,
  arc=5pt,
  leftrule=0pt,
  rightrule=0pt,
  bottomrule=2pt,
  toprule=2pt,
  colback=bg,
  colframe=orange!70,
  enhanced,
  overlay={%
    \begin{tcbclipinterior}
    \fill[orange!20!white] (frame.south west) rectangle ([xshift=20pt]frame.north west);
    \end{tcbclipinterior}},
  #3}
%---------------------------------------------

%------
% Font
%------
\pagenumbering{Roman}
%\usepackage[T1]{fontenc}
\usepackage[utf8]{inputenc} % allows non-ascii input characters
\usepackage{tikz-feynman}
% Comment or remove the two lines below to use the default Computer Modern font
%\usepackage{libertine}
%\usepackage{libertinust1math}
\usepackage{csvsimple}

%\usepackage[sc]{mathpazo} %font or smth
\linespread{1.05}
%\usepackage{microtype}

\usepackage{fancyhdr} % Headers and footers
\pagestyle{fancy} % All pages have headers and footers e.g contents above contents (except new ones)
\usepackage[Rejne]{fncychap} %fancy chapters
%options for chapters: Sonny, Lenny, Glenn, Conny, Rejne, Bjarne, Bjornstrup

% NOTE: because the UGent font Panno is proprietary, it is not possible to use it
% in Overleaf. But UGent does not suggest to use Panno for documents (or maybe only for
% the titlepage). For the body, the UGent suggestion is to use a good serif font (for
% LaTeX this could be libertine or Computer Modern).
\usepackage{slashed}
\usepackage{braket}

% Proper word splitting
%-----------------------
\usepackage[english]{babel} 

% Mathematics
%-------------
\usepackage{amsmath}
\usepackage{amsfonts}
\usepackage{mathrsfs}
% Figures
%---------
%\usepackage{graphicx} % optional: the package is already loaded by the template
\graphicspath{{./figures/}}
\usepackage{copyrightbox}
% Bibliography settings
%-----------------------
\usepackage{cite}

% Hyperreferences
%-----------------
\usepackage[colorlinks=true, allcolors=ugentblue]{hyperref}

% Whitespace between paragraphs and no indentation
%--------------------------------------------------
\usepackage[parfill]{parskip} 


