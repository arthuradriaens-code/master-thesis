\chapter*{Proposal for improved measurements}
\addcontentsline{toc}{chapter}{Proposal for improved measurements}
As this method seems to be a viable way of measuring the index of refraction,
it might be a good idea to have a more controllable radio wave source fly
closer to the detectors to make more accurate measurements, e.g an
autonomous\footnote{as to not have it need a radio controller, causing RF
interference and also as autonomous GNSS positioning is always more
accurate than human steering} drone with an antenna strapped to it.  Assuming Schytt's
equation to hold completely, ideally we'd like data inbetween the depths of
20-100m as this is where the biggest discrepencies might be observed between
the single exponential model and the measured density data as is depicted on
figure \ref{fig:DensityMeasurements}. 

In the super ideal case we'd want some unmoving close source and a phased array which we'll
move down every couple of measurements, thus obtaining an index-depth profile
using direct measurements.

