\documentclass[11pt,a4paper,faculty=we,language=en,doctype=report]{cls/ugent-doc}

% Optional: margins and spacing
%-------------------------------
% Uncomment and adjust to change the default values set by the template
% Note: the defaults are suggested values by Ghent University
%\geometry{bottom=2.5cm,top=2.5cm,left=3cm,right=2cm} 
%\renewcommand{\baselinestretch}{1.15} % line spacing

% Font
%------
%\usepackage[T1]{fontenc}
\usepackage[utf8]{inputenc} % allows non-ascii input characters
% Comment or remove the two lines below to use the default Computer Modern font
%\usepackage{libertine}
%\usepackage{libertinust1math}

\usepackage[sc]{mathpazo} %font or smth
\linespread{1.05}
\usepackage{microtype}

\usepackage{fancyhdr} % Headers and footers
\pagestyle{fancy} % All pages have headers and footers e.g contents above contents (except new ones)
\usepackage[Rejne]{fncychap} %fancy chapters
%options for chapters: Sonny, Lenny, Glenn, Conny, Rejne, Bjarne, Bjornstrup

% NOTE: because the UGent font Panno is proprietary, it is not possible to use it
% in Overleaf. But UGent does not suggest to use Panno for documents (or maybe only for
% the titlepage). For the body, the UGent suggestion is to use a good serif font (for
% LaTeX this could be libertine or Computer Modern).

% Proper word splitting
%-----------------------
\usepackage[english]{babel} 

% Mathematics
%-------------
\usepackage{amsmath}

% Figures
%---------
%\usepackage{graphicx} % optional: the package is already loaded by the template
\graphicspath{{./figures/}}

% Bibliography settings
%-----------------------
\usepackage{cite}

% Hyperreferences
%-----------------
\usepackage[colorlinks=true, allcolors=ugentblue]{hyperref}

% Whitespace between paragraphs and no indentation
%--------------------------------------------------
\usepackage[parfill]{parskip} 

% Input for title page
%----------------------

% The title
\thetitle{Radio detection of high energy neutrinos in the Greenland icecap}
\thesubtitle{Arthur Adriaens}

%% Note: a stricter UGent style could be achieved with, e.g.:
\usepackage{ulem} % for colored underline
\renewcommand{\ULthickness}{2pt} % adjust thickness of underline
\thetitle{\uline{\color{ugentblue}Radio detection of high energy neutrinos in the Greenland icecap}}
% Note: do not forget to reset the \ULthickness to 1pt after invoking \maketitle
% (otherwise all underlines in the rest of your document will be too thick):
%\renewcommand{\ULthickness}{1pt}

% The first (top) infobox at bottom of titlepage
\infoboxa{\bfseries\large Department of Physics and Astronomy}

% The second infobox at bottom of titlepage
\infoboxb{Promotor: 
\begin{tabular}[t]{ll}
    Prof. dr. Dirk Ryckbosch & Dirk.Ryckbosch@ugent.be\\ % note syntax 'short space'
\end{tabular}
}

% The third infobox at bottom of titlepage
\infoboxc{Accompanist: 
\begin{tabular}[t]{ll}
    Bob Oeyen &  Bob.Oeyen@ugent.be\\
\end{tabular}
}

% The last (bottom) infobox at bottom of titlepage
\infoboxd{Academic year: 2022--2023} % note dash, not hyphen
\infoboxd{Master’s dissertation submitted in partial fulfilment of the requirements for the degree of master in Physics and Astronomy}

\begin{document}

% =====================================================================
% Cover
% =====================================================================

% ------------ TITLE PAGE ---------
\maketitle
\renewcommand{\ULthickness}{1pt}

% =====================================================================
% Front matter
% =====================================================================

% ------------ TABLE OF CONTENTS ---------
{\hypersetup{hidelinks}\tableofcontents} % hide link color in toc
\newpage


% =====================================================================
% Main matter
% =====================================================================
\chapter{Neutrinos}
\section{Discovery}
\section{Standard model}
\section{Outside sources}
\subsection{Cosmic neutrinos}
To estimate the temperature of the neutrinos who decoupled at the start of the universe, we can take a look at conservation of entropy \cite{Dodelson}
(...)
The entropy before and after decoupling are:
\begin{align}
	s(a_1) &= \frac{2\pi^2}{45}(2 + \frac{7}{8}(2+2+3+3))T_1^3\\
	&= \frac{2\pi^2}{45}\frac{86}{8}T_1^3\\
	s(a_2) &= \frac{2\pi^2}{45}(2T_\gamma^3 + \frac{7}{8}(6)T_\nu^3)\\
\end{align}
Conservation of entropy:
\begin{align}
	s(a_1)a_1^3 &= s(a_2)a_2^3\\
	\frac{86}{8}(T_1 a_1)^3 &= \left(2\left(\frac{T_\gamma}{T_\nu}\right)^3 + \frac{42}{8}\right)(T_\nu a_2)^3\\
	\frac{86}{8} &= 2\left(\frac{T_\gamma}{T_\nu}\right)^3 + \frac{42}{8}\\
	\frac{44}{16} &= \left(\frac{T_\gamma}{T_\nu}\right)^3\\
	\left(\frac{T_\gamma}{T_\nu}\right) &= \left(\frac{11}{4}\right)^{1/3}
\end{align}
i.e
\begin{equation}
	T_\nu = \left(\frac{4}{11}\right)^{1/3}T_\gamma
\end{equation}
$\Phi$
\subsection{Oscillations}
\subsection{Majorana}
\newpage
\chapter{Radio detection}


% =====================================================================
% End matter
% =====================================================================
\newpage
%----------------------------------------------------------------------------------------
%	REFERENCE LIST
%----------------------------------------------------------------------------------------
\bibliography{sources}
\bibliographystyle{plain}

\end{document}
