\shipout\null
\newpage
\chapter*{Abstract}
\addcontentsline{toc}{chapter}{Summary}
To infer information about the cosmos, the neutrino is an ideal candidate as it
points back to the event itself.  Due to the neutrino not having any charge,
nearly no mass and interacting weakly we can be quite certain that if we
observe a neutrino back here on earth it came from the direction we found it in.

There are, and have been, numerous neutrino detectors. But there was a need to 
build another one as \textit{cosmogenic} neutrinos hadn't been observed jet.
cosmogenic neutrinos are very high in energy but consequently are very low in flux,
to observe cosmogenic neutrinos in human lifetimes you'll thus need a very big detector.
The economical and logistical choice is to build a detector in the Greenland icecap
on the principle of radio waves called \textit{RNO-G}, for Radio Neutrino Observatory
in Greenland.

As this detector is built in an icecap and the principle of detecting neutrinos is reliant
on the physics of how the waves propagate through the ice, it is necessary to understand 
the optical properties of the Greenland ice cap well. An important optical property
for neutrino reconstruction is the index of refraction-depth relation which, if modelled,
is called an \textit{ice model}.

The ice model currently being used is largely based on density-depth data but is 
a sub-ideal fit, in this thesis an algorithm was built to make use of a more complex
ice model in chapter \ref{chapter:hybrid} and to test the difference between the
used ice model and the actual ice properties, balloon plane wave reconstructions were
performed which were used to map out the index-depth profile, as explained in chapter \ref{chap:WB}.

The following data was obtained this way:
\begin{center}
\begin{tabular}{||c c c c c c||}
 \hline
 Depth (m) & Station id & channels & Run:Event & n$_\text{exponential}$ & n$_\text{fit}$\\ [0.5ex]
 \hline\hline
 -48.155 & 21 & 6\&7 & 1441:117 & 1.6400 & 1.632 $\pm$ 0.00128 $\pm$ 0.00472\\
 -58.38 & 21 & 5\&7 & 1441:117 & 1.6736 & 1.66632 $\pm$ 0.00226 $\pm$ 0.00233 \\
 -58.24 & 21 & 5\&6\&7 & 1441:117 & 1.67321 & 1.66553 $\pm$ 0.00167 $\pm$ 0.00350 \\
 -68.2 & 21 & 5\&6 & 1441:117 & 1.6983 & 1.69329 $\pm$0.00199$\pm$0.00460 \\
 -93.865 & 21 & 0\&1\&2\&3 & 1441:117 & 1.73896 & 1.71366$\pm$0.00287$\pm$0.05641\\
 -94.518 & 11 & 0\&2\&3 & 1034:12397 & 1.7397 & 1.724 $\pm$ 0.014 $\pm$ 0.047 \\
 \hline
\end{tabular}
\end{center}
\newpage
\chapter*{Samenvatting}
Om informatie over het heelal te leren is de neutrino een bijzonder goede bron aangezien
ze, door haar zwakke interactiekracht, ongehinderd naar de aarde propageert.
Als we dus een neutrino op aarde detecteren en de richting kunnen infereren zijn we
vrijwel zeker waar ze vandaan komt.

Wegens deze handige eigenschappen van de neutrino zijn er al verscheidene neutrinodetectoren gebouwd.
Maar er is een nood voor de bouw van nog een detector aangezien cosmogenische neutrinos nog niet
geobserveerd zijn. Dit zijn neutrinos met zeer hoge energiën en dus ook laag in aantal,
om deze neutrinos te kunnen observeren in doenbare tijdsperiodes is dus een grote detector nodig.
Om deze reden werd de \textit{Radio Neutrino Observatory in Greenland} of RNO-G gebouwd, een 
neutrino detector gebaseerd in het Groenlandse ijs.

Aangezien deze detector gebouwd is in het Groenlandse ijs en deze gebaseerd is
op het principe van daarin gepropageerde radio golven is het cruciaal de
optische eigenschappen van het ijs te kennen. Een zeer belangrijke optische
eigenschap is the index van refractie-diepte relatie, dewelke gemodelleerd word
en dan een \textit{ijsmodel} wordt genoemd.

Het momentaan gebruikte ijsmodel is gebaseerd op een fit aan een lineaire omzetting van dichtheid-diepte metingen.
Deze fit kan sub-optimaal zijn en om accuraat met moeilijkere ijsmodellen
te kunnen omgaan werd in deze thesis, in hoofdstuk \ref{chapter:hybrid}, een nieuw \textit{ray tracing}
algoritme gebouwd. Om te kijken of het wel degelijk nodig is een nieuwe te gebruiken werden 
metingen uitgevoerd aan de hand van gedetecteerde weerballon signalen, zoals uitgelegd in hoofdstuk \ref{chap:WB}. 
De bekomen data wordt hieronder weergegeven:
\begin{center}
\begin{tabular}{||c c c c c c||}
 \hline
 Depth (m) & Station id & channels & Run:Event & n$_\text{exponential}$ & n$_\text{fit}$\\ [0.5ex]
 \hline\hline
 -48.155 & 21 & 6\&7 & 1441:117 & 1.6400 & 1.632 $\pm$ 0.00128 $\pm$ 0.00472\\
 -58.38 & 21 & 5\&7 & 1441:117 & 1.6736 & 1.66632 $\pm$ 0.00226 $\pm$ 0.00233 \\
 -58.24 & 21 & 5\&6\&7 & 1441:117 & 1.67321 & 1.66553 $\pm$ 0.00167 $\pm$ 0.00350 \\
 -68.2 & 21 & 5\&6 & 1441:117 & 1.6983 & 1.69329 $\pm$0.00199$\pm$0.00460 \\
 -93.865 & 21 & 0\&1\&2\&3 & 1441:117 & 1.73896 & 1.71366$\pm$0.00287$\pm$0.05641\\
 -94.518 & 11 & 0\&2\&3 & 1034:12397 & 1.7397 & 1.724 $\pm$ 0.014 $\pm$ 0.047 \\
 \hline
\end{tabular}
\end{center}

\addcontentsline{toc}{chapter}{Samenvatting}
\newpage
% ------------ TABLE OF CONTENTS ---------
{\hypersetup{hidelinks}\tableofcontents} % hide link color in toc
\newpage


