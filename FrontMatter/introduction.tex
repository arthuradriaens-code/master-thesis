\chapter*{Introduction}
\addcontentsline{toc}{chapter}{Introduction}
\pagenumbering{arabic}
Outside our earth various kinds of events take place which we wish to 
observe: Black hole outbursts, supernovae, cosmic jets, ...
These events produce various kinds of messengers like 
Gravitational waves, gamma rays, protons,...
But one of these information carriers is unique and
the subject of our study: The neutrino. 

The neutrino has the special property that it points back to the event itself.
Due to the neutrino not having any charge, nearly no mass and 
interacting weakly it doesn't get bent nor absorbed and re-emitted 
on it's way to us unlike say the proton. This means that if we observe
a neutrino back here on earth it's very likely that the direction we observe
it in, is the direction it came from, the neutrino is explained more in-depth in chapter
\ref{chap:neutrino}.

The neutrinos in the ultra high energy (UHE) range are of high importance to us.
There have been lots of neutrino detectors around but none of them
have been able to observe \textit{cosmogenic neutrinos} which would have energies above
the 3 PeV energy range. We probably haven't detected them as the flux falls exponentially 
with increasing energy, implying that a really big detector volume is
required to detect neutrinos in reasonable timespans. 

Both economically and logistically it makes sense to build a detector of the
required size in ice and let it operate in the radio regime, as was previously
explored in experiments like ARA and ARIANNA.  To this end the Radio Neutrino
Observatory in Greenland or RNO-G is currently under construction and how it 
detects neutrinos is explained in chapter \ref{chap:RND}.

The ice properties have an impact on how radio waves propagate. As mentioned
before, RNO-G is a detector built in the Greenland firn. As the radio waves get
produced in the ice we need to figure out how they propagate towards our
detector. An important part in figuring out how they propagate is understanding
the optical properties of the ice with a big factor being the local index of
refraction. This index of refraction seems to be linearly related to the density of the ice which
varies continuously with depth. So wee need a model describing the overall index-depth relation
called an \textit{ice model}, how this is used to understand how the rays propagate
is explained in chapter \ref{chap:RT}.

The ice model currently used seems to be a sub-optimal fit to the index-depth
data (as converted from the density-depth data).  To both work with more
complex models and convince people to work with those models, some work needs
to be done.  

In this thesis, in chapter \ref{chapter:hybrid}, a new algorithm was developed
which makes it possible to work faster and more accurate with complex ice
models whom might follow the real index-depth profile closer. And as to try and
show the discrepency of the real index-depth relation to the model now widely
used, plane wave reconstruction of observed weather balloon signals was carried
out in \ref{chap:WB}.
